
\chapter{Make Index}
\label{make-index}

Making INDEX\index{index} for the document is really a mess. The steps
are as follws:

\begin {itemize}
   \item label indices in the text files
   \item insert a makeindex command in {\tt thesis.tex}
   \item run {\tt thesis.tex} to generate {\tt thesis.ind}
   \item remove makeindex command
   \item run a third party program to make {\tt thesis.idx}
   \item input {\tt thesis.idx} to {\tt thesis.tex}
   \item run {\tt thesis.tex} to produce mighty {\bf INDEX} in the document.
\end{itemize}
\section{Now in Detail}

I understand that you have read \cite[pp:77--79]{lamport86}.

It is really important that you wait to prepare the index\index{index}
until you are completely ready to print it on the nice 25\% cotton
bond paper to submit to the graduate office. If you still have to add
some text in the chapters or you have to add some figures the final
output of page numbering may change, and that is why I recommend that
you prepare the index in your leisure time.

Labels are marked in the following way: if, for example, you need to
add a word {\tt ``label''}\index{label} in the index, then you need to
use the $backslash$index command after the apperance of the world {\tt
``label''} in the text: {\tt label$\backslash$index\{label\}}.

I would suggest you that don't mark index labels in the original
document; rather, make a copy of all chapters in one file called {\tt
all-chapters.tex}. The reason for this is that, if you have more than
one chapter and you want to put index label for a word that is used in
more than one chapter, then it will be time consuming to bring each
chapter in the editor separately and adding labels. If you have all
chapters in one temporary file, it is easier to give a command to
change the word from {\tt xyz} to {\tt xyz$\backslash$index\{xyz\}}. I
warn you not to do this with a global command because it may change
some words which you don't want to be labeled. For example, suppose
you want to put the word {\tt ``label''} in the index\index{index} for
the following text:

\begin{verse}
{\tt Indexing is not really an easy job. It requires
\underline{label}ing the words in the text. For \underline{label}ing
you must have some motivation for \underline{label}ization your
document. You need a lot of patients. Including global
\underline{label} is harmful.}
\end{verse}

In the above example all underline words will be labeled. However,
you may not want a index label on the segment of the word.

Here are some steps I did for preparing an index on my document 
	\cite{soomro94}.
\begin{enumerate}
   \item copy all chapters to a {\tt all-chapters-for-index.tex}\\
	I did not copy all chapters to one file while preparing this document.
   \item put index\index{index} labels in the file 
		{\tt all-chapters-for-index.tex}
   \item insert a command {\tt $\backslash$makeindex} or just remove\\
	\% comment before it in the file thesis-sample.tex
   \item put \% comment before all chapter inputs in your main thesis.tex\\
	in this case thesis-sample.tex
   \item add a command {\tt $\backslash$input all-chapters-for-index.tex}
   \item run the thesis.tex throught \LaTeX
   \item make sure that the {\tt thesis.idx} file is generated
   \item remove {\tt $\backslash$makeindex} command or put a \% for comment 
		before it in the {\tt thesis.tex} file
   \item remove {\tt $\backslash$input all-chapters-for-index.tex}
   \item remove comments sign \% before all chapter inputs in the
		{\tt thesis.tex} file
   \item using a third party program, which I have here as 
		{\tt makeindex-sun.exe} and {\tt makeindex-sparc.exe}   
   \item at prompt, type {\tt makeindex-sun.exe thesis.idx}\\
		(You may ignore ``.idx'')
   \item you should have the {\tt thesis.ind} file generated
   \item go back to your {\tt thesis.tex} file and add 
		{\tt $\backslash$input thesis.ind} before the
	{\tt $\backslash$end\{document\}} command
   \item you may edit {\tt thesis.ind} to make items or subitems as you wish
   \item run {\tt thesis.tex} twice, or three times to be on the safe side
   \item now you have index in your document
\end{enumerate}
